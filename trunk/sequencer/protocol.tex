\documentclass[10pt, a4paper, parskip]{scrartcl}
\usepackage[latin1]{inputenc}
\usepackage[T1]{fontenc}
\usepackage{booktabs}
\usepackage{svn}
\usepackage{pslatex}
\usepackage[obeyspaces]{url}    % Das Paket Hyperref �berschreibt \url,
    \let\urlurl=\url            % also wird es hier gesichert. (\urlurl
                                % ist url's Version von \url.)
\usepackage{hyperref}           % Muss wegen schwarzer Magie als
                                % letztes Paket erscheinen.

\parskip=4pt plus 2pt           % Die Option parskip sorgt f�r Abs�tze ohne
                                % Einr�ckung, aber sehr gro�em
                                % Abstand. Das wird hier korrigiert.

\setcounter{secnumdepth}{1}     % Sections erhalten Nummern, feinere
                                % Unterteilungen aber nicht.

\let\code=\urlurl            % Codesegmente werden wie URLs formatiert
\def\url#1{\href{#1}{\urlurl{#1}}}

%% Formattab: eine Tabelle zum Beschreiben von Datenformaten
\let\scf=\sectfont              % Fettdruck wie in den �berschriften.
\newenvironment{commandtab}[1]%
{\noindent
  \begin{tabular*}{\linewidth}{@{\extracolsep\fill}llp{0.5\linewidth}}
    \multicolumn2c{\scf#1}\\ \addlinespace[0.2ex]
    \scf Name &\scf Description\\
  }%
{\end{tabular*}
\medskip}

\hyphenation{ur-met sen-der}

\hypersetup{
  colorlinks=true,
  linkcolor=blue,
  citecolor=blue,
  pdftitle={The Villa Protocol}}

\begin{document}

\SVN $LastChangedDate: 2005-08-28 22:45:17 +0200 (Sun, 28 Aug 2005) $
\SVN $LastChangedRevision: 18 $

\author{Lars Immisch}
\title{The Villa Protocol}
\date{\SVNDate\thanks{Revision \SVNLastChangedRevision}}
\maketitle

\begin{abstract}
  \noindent
  The \emph{Villa} is a Telephone-based multi-user dungeon.
  This document describes the \emph{Villa protocol} that is spoken between the 
  game logic and an audio renderer. The protocol is a simple, text oriented 
  protocol on top of TCP, inspired by IMAP. It provides sophisticated 
  operations for the realtime manipulation of audio playlists and basic call 
  control.
\end{abstract}

\section{Overview}
\label{sec:overview}

The protocol has eight command verbs:

\begin{itemize}
\item \verb+LSTN+ (listen)
\item \verb+DISC+ (disconnect)
\item \verb+ACPT+ (accept)
\item \verb+MLCA+ (molecule add)
\item \verb+MLCD+ (molecule discard)
\item \verb+MLDP+ (molecule discard by priority)
\item \verb+CNFO+ (conference open)
\item \verb+CNFC+ (conference close)
\end{itemize}

And four event verbs:

\begin{itemize}
\item \verb+DTMF+ (DTMF detected)
\item \verb+RDIS+ (Remote disconnect)
\item \verb+ABEG+ (Atom beginning to run)
\item \verb+AEND+ (Atom ending)
\end{itemize}

A message is either a request, a completion or an event.

\subsection{Basic structure of a request}

The basic structure of a request is:

\emph{tid device verb arguments}

\emph{tid} is a transaction id. Most often, it will be a number, but it can 
also be alphanumerical. The string \verb+-1+ is reserved and must not be used
(it is used for unsolicited events).

Transactions IDs are used to correlate a completion to a request.

\emph{device} is a name picked by the rendering engine. It uniquely identifies 
a channel, normally a telephone line and associated speech processing 
capabilities.

\emph{verb} is one of the command verbs above.

\emph{arguments} depends on \emph{verb} and is a list of arguments to the
request, separated by whitespace.

A simple example of a request is:
\begin{verbatim}
0 GLOBAL LSTN any any
\end{verbatim}

This instructs the audio renderer to wait for an incoming call on any trunk 
and any number.

When a call comes in, the audio renderer will respond with:

\begin{verbatim}
0 ETS300[0:3] LSTN 
\end{verbatim}


\section{Molecules}


\begin{thebibliography}{S.200}  % Die Breite dieses Parameters bestimmt,
                                % wie weit Folgezeilen eines
                                % Bibliographieeintrags einger�ckt werden.
  \rightskip=0pt plus 4em       % Etwas Dehnbarkeit am rechten Rand
                                % wegen der langen URLs; \raggedright
                                % w�re zu nachgiebig.
  
\bibitem[ECTF]{ectf-specs} ECTF -- Specifications, Published Documents
  Online,
  \url{http://www.comptia.org/sections/ectf/specifications.aspx}.

\bibitem[M.100]{m100} 
  M.100, ECTF Administration services API,
  \url{http://www.comptia.org/sections/ectf/Documents/m100r1-0.pdf}.

\bibitem[Reactor pattern]{reactor}
  The Reactor pattern,
  \url{http://www.cs.wustl.edu/~schmidt/PDF/reactor-siemens.pdf}.
  
\bibitem[RFC 2373]{rfc2373} RFC 2373 -- IP Version 6 Addressing
  Architecture, \url{http://www.faqs.org/rfcs/rfc2373.html}.

\bibitem[RFC 2396]{rfc2396} RFC 2396 -- Uniform Resource Identifiers
  (URI): Generic Syntax, \url{http://www.faqs.org/rfcs/rfc2396.html}.

\bibitem[S.100]{s100}
  S.100, ECTF Media Services API,
  \url{http://www.comptia.org/sections/ectf/Documents/s100r2.zip}.
  
\bibitem[S.100r2v4]{s100r2v4} S.100 Media Services, Volume 4:
  Framework, revision 2.0. Located at \cite{ectf-specs}.
  
\bibitem[S.200]{s200}
  S.200, ECTF Media Services Protocol Specification,
  \url{http://www.comptia.org/sections/ectf/Documents/s200r0-1.pdf}.
  
\bibitem[S.300v1r1]{s300v1r1} ECTF Media Services SPI Specification,
  Volume 1: Overview, revision 1.0. Located at \cite{ectf-specs}.

\bibitem[S.300]{s300}
  S.300, ECTF Service Provider Interface Specification,
  \url{http://ibp.de/ECTF/S.300/html/}.
  
\bibitem[S.410]{s410r2} S.410 JTAPI Media Revision 2. Located at
  \cite{ectf-specs}.

\end{thebibliography}
\end{document}